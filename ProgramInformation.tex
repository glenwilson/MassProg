\documentclass{article}

% Motivic Adams Spectral Sequence Calculator
% Copyright (C) 2016 Glen Matthew Wilson

% This program is free software; you can redistribute it and/or modify
% it under the terms of the GNU General Public License as published by
% the Free Software Foundation; either version 2 of the License, or (at
% your option) any later version.

% This program is distributed in the hope that it will be useful, but
% WITHOUT ANY WARRANTY; without even the implied warranty of
% MERCHANTABILITY or FITNESS FOR A PARTICULAR PURPOSE.  See the GNU
% General Public License for more details.

% You should have received a copy of the GNU General Public License
% along with this program; if not, write to the Free Software
% Foundation, Inc., 51 Franklin Street, Fifth Floor, Boston, MA
% 02110-1301 USA.

\begin{document}

\title{Motivic Adams spectral sequence program}
\author{Knight Fu}
\author{Glen M. Wilson}
\maketitle


\section{Introduction}

\section{Basic operation of program}
The most basic operation of the program using the wrapper class to
make the commands simple to enter. Create a new python script file and
start by importing the classes and functions from the wrapper file.

\begin{quote}
\begin{verbatim}
from wrapper import *
\end{verbatim}
\end{quote}

The program will produce the $E_2$ page of either the motivic Adams
spectral sequence over certain fields $k$, or of the Adams spectral
sequence in topology. You must declare which case you would like to
calculate by modifying the program options. The default options shold
be good enough to get you started. You can choose the default options
for calculation the motivic Adams spectral sequence over the real
numbers with the following.
\begin{quote}
\begin{verbatim}
options = Options.default("Real")
\end{verbatim}
\end{quote}

We then declare our spectral sequence object by entering the
following.
\begin{quote}
\begin{verbatim}
myss = MASS(options)
\end{verbatim}
\end{quote}
You can then start the MASS calculation with the following command.
\begin{quote}
\begin{verbatim}
myss.start_session()
\end{verbatim}
\end{quote}

This program calculates the $E_2$ page of the Adams spectral sequence
by constructing a minimal resolution of $H^{**}$ by $\mathcal{A}^{**}$
modules, then dualize and take cohomology. We first ask the program to
calculate the minimal resolution. 
\begin{quote}
\begin{verbatim}
myss.make_no_mat_resolution()
\end{verbatim}
\end{quote}
You can now end your session with 
\begin{quote}
\begin{verbatim}
myss.stop_session()
\end{verbatim}
\end{quote}
which will save the minimal resolution you just calculated. You can
now try running your script file. With the default options, you should
see several new files in your working directory \texttt{real\_*}. 

The file \texttt{real\_log.log} will store a great deal of information
about the calculation. If you followed the above directions, this file
will contain information on the construction of the minimal resolution
$\cdots \to P_2 \to P_1 \to P_0 \to H^{**}$. The program starts off
with the evident map $P_0 = \mathcal{A}^{**} \to H^{**}$ and sets
$P_1 = 0$ to start. Then by checking the kernel of $P_0 \to H^{**}$
and the image of $P_1 \to P_0$ in each degree, it will add generators
to $P_1$ as necessary until it is exact in each degree.

The log file shows the program starts with the degree 0 weight 0 part
of the first map (corresponding to stem -1, weight 0, filtration 1)
and observes that the kernel is 0 dimensional, and the map
$P_1 \to P_0$ as defined thus far has 0 dimensional image in this
graded piece. So the program doesn't need to add a new generator to
$P_1$ yet.

We see that the next graded piece is $(1,0)$. Here the kernel is 1
dimensional, and the image is 0 dimensional, so a generator must be
added to $P_1$. The log file shows a new generator is needed, and
describes the name given to it \texttt{h1(1,0)0}. The general naming
scheme of generators is \texttt{hf(d,w)n} where $f$ indicates the
generator is added to $P_f$ in the minimal resolution, $d$ is the
degree, $w$ is the weight, and $n$ is used to distinguish between
generators in the same filtration, degree, and weight.

The next interesting point in the log file is at the graded piece
$(2,1)$. Here we expect the class $h_1$ to make an appearance. Indeed
it does, and the internal program name for this class is
\texttt{h1(2,1)0}.

Writing all of this information to the log file does slow the program
down. You can turn off logging with the command
\begin{quote}
\texttt{myss.set\_logging\_level(logging.ERROR)}
\end{quote}
or increase the amount of logging with logging level
\texttt{logging.DEBUG}. You can return the logging level to default
with the level \texttt{logging.WARNING}.

To calculate the $E_2$ page of the MASS, we still need to dualize this
minimal resolution and calculate its cohomology and product structure.
This can be accomplished with the following commands. 
\begin{quote}
\begin{verbatim}
myss.make_dual_resolution()
myss.compute_product_structure()
\end{verbatim}
\end{quote}
The $E_2$ page of the MASS has now been calculated. There are several
options available to saving and presenting this information. The first
is to save the vector space and product structure to a MYSQL
database. Insert the details of your MySQL database into the file
\texttt{./db/etwo.py}. The structure of the $E_2$ page can now be
saved to your database with the
following command.
\begin{quote}
\begin{verbatim}
myss.make_product_database()
\end{verbatim}
\end{quote}

You can also produce charts of the $E_2$ page with the help of the
\texttt{matplotlib} library. There are several plot options available.
The default plots can be obtained with the following command.
\begin{quote}
\begin{verbatim}
myss.make_charts()
\end{verbatim}
\end{quote}

The commands \texttt{myss.make\_resolution(),
  myss.make\_no\_mat\_resolution(), myss.make\_dual\_resolution()}
only need to be run once for a particular configuration. Running them
again will just waste a lot of time, so comment them out of your
script after they have been run once. If you change the bounds, you'll
need to run these commands again.

As it currently stands, you must run
\texttt{myss.compute\_product\_structure()} in each session if a
command will depend on the product structure, e.g.,
\texttt{make\_product\_database}.


\section{Nuts and bolts}

\end{document}