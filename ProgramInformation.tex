\documentclass{article}

% Motivic Adams Spectral Sequence Calculator Copyright (C) 2016 Glen
% Matthew Wilson

% This program is free software; you can redistribute it and/or modify
% it under the terms of the GNU General Public License as published by
% the Free Software Foundation; either version 2 of the License, or
% (at your option) any later version.

% This program is distributed in the hope that it will be useful, but
% WITHOUT ANY WARRANTY; without even the implied warranty of
% MERCHANTABILITY or FITNESS FOR A PARTICULAR PURPOSE.  See the GNU
% General Public License for more details.

% You should have received a copy of the GNU General Public License
% along with this program; if not, write to the Free Software
% Foundation, Inc., 51 Franklin Street, Fifth Floor, Boston, MA
% 02110-1301 USA.
\newcommand{\Sq}{\mathrm{Sq}}
\newcommand{\calA}{\mathcal{A}}

\usepackage{amssymb,amsmath}
\begin{document}

\title{Motivic Adams spectral sequence program} \author{Knight Fu \and
  Glen M. Wilson}

\maketitle

\section{Basic operation of program}
\subsection{Charts and product database}

The most basic operation of the program using the wrapper class to
make the commands simple to enter. Create a new python script file and
start by importing the classes and functions from the wrapper file.

\begin{quote}
\begin{verbatim}
from wrapper import *
\end{verbatim}
\end{quote}

The program will produce the $E_2$ page of either the motivic Adams
spectral sequence over certain fields $k$, or of the Adams spectral
sequence in topology. You must declare which case you would like to
calculate by modifying the program options. The default options shold
be good enough to get you started. You can choose the default options
for calculation the motivic Adams spectral sequence over the real
numbers with the following.
\begin{quote}
\begin{verbatim}
options = Options.default("Real")
\end{verbatim}
\end{quote}

We then declare our spectral sequence object by entering the
following.
\begin{quote}
\begin{verbatim}
myss = MASS(options)
\end{verbatim}
\end{quote}
You can then start the MASS calculation with the following command.
\begin{quote}
\begin{verbatim}
myss.start_session()
\end{verbatim}
\end{quote}

This program calculates the $E_2$ page of the Adams spectral sequence
by constructing a minimal resolution of $H^{**}$ by $\mathcal{A}^{**}$
modules, then dualize and take cohomology. We first ask the program to
calculate the minimal resolution.
\begin{quote}
\begin{verbatim}
myss.make_no_mat_resolution()
\end{verbatim}
\end{quote}
You can now end your session with
\begin{quote}
\begin{verbatim}
myss.stop_session()
\end{verbatim}
\end{quote}
which will save the minimal resolution you just calculated. You can
now try running your script file. With the default options, you should
see several new files in your working directory \texttt{real\_*}.

The file \texttt{real\_log.log} will store a great deal of information
about the calculation. If you followed the above directions, this file
will contain information on the construction of the minimal resolution
$\cdots \to P_2 \to P_1 \to P_0 \to H^{**}$. The program starts off
with the evident map $P_0 = \mathcal{A}^{**} \to H^{**}$ and sets
$P_1 = 0$ to start. Then by checking the kernel of $P_0 \to H^{**}$
and the image of $P_1 \to P_0$ in each degree, it will add generators
to $P_1$ as necessary until it is exact in each degree.

The log file shows the program starts with the degree 0 weight 0 part
of the first map (corresponding to stem -1, weight 0, filtration 1)
and observes that the kernel is 0 dimensional, and the map
$P_1 \to P_0$ as defined thus far has 0 dimensional image in this
graded piece. So the program doesn't need to add a new generator to
$P_1$ yet.

We see that the next graded piece is $(1,0)$. Here the kernel is 1
dimensional, and the image is 0 dimensional, so a generator must be
added to $P_1$. The log file shows a new generator is needed, and
describes the name given to it \texttt{h1(1,0)0}. The general naming
scheme of generators is \texttt{hf(d,w)n} where $f$ indicates the
generator is added to $P_f$ in the minimal resolution, $d$ is the
degree, $w$ is the weight, and $n$ is used to distinguish between
generators in the same filtration, degree, and weight.

The next interesting point in the log file is at the graded piece
$(2,1)$. Here we expect the class $h_1$ to make an appearance. Indeed
it does, and the internal program name for this class is
\texttt{h1(2,1)0}.

Writing all of this information to the log file does slow the program
down. You can turn off logging with the command
\begin{quote}
  \texttt{myss.set\_logging\_level(logging.ERROR)}
\end{quote}
or increase the amount of logging with logging level
\texttt{logging.DEBUG}. You can return the logging level to default
with the level \texttt{logging.WARNING}.

To calculate the $E_2$ page of the MASS, we still need to dualize this
minimal resolution and calculate its cohomology and product structure.
This can be accomplished with the following commands.
\begin{quote}
\begin{verbatim}
myss.make_dual_resolution()
myss.compute_product_structure()
\end{verbatim}
\end{quote}
The $E_2$ page of the MASS has now been calculated. There are several
options available to saving and presenting this information. The first
is to save the vector space and product structure to a MYSQL
database. Insert the details of your MySQL database into the file
\texttt{./db/etwo.py}. The structure of the $E_2$ page can now be
saved to your database with the following command.
\begin{quote}
\begin{verbatim}
myss.make_product_database()
\end{verbatim}
\end{quote}

You can also produce charts of the $E_2$ page with the help of the
\texttt{matplotlib} library. There are several plot options available.
The default plots can be obtained with the following command.
\begin{quote}
\begin{verbatim}
myss.make_charts()
\end{verbatim}
\end{quote}

The commands \texttt{myss.make\_resolution(),
  myss.make\_no\_mat\_resolution(), myss.make\_dual\_resolution()}
only need to be run once for a particular configuration. Running them
again will just waste a lot of time, so comment them out of your
script after they have been run once. If you change the bounds, you'll
need to run these commands again.

As it currently stands, you must run
\texttt{myss.compute\_product\_structure()} in each session if a
command will depend on the product structure, for example,
\texttt{make\_product\_database}.

\subsection{Changing bounds}

If you ran the above with the default options, the program will stop
once it gets to degree 12 and weight 6. Should you wish to calculate
beyond these bounds, you need to declare this in the options
variable. Use the following command to do so.
\begin{quote}
\begin{verbatim}
myss.set_degree_bounds((18, 9))
\end{verbatim}
\end{quote}
The bounds should be of the form $(2n, n)$ for a natural number $n$,
although it isn't strictly necessary.

Once you reset the bounds, you'll need to generate the resolution,
dual resolution, and product structures again. The program is able to
use the existing resolution information and just add on to it. This
cuts down on duplicated calculations.

\subsection{Massey products}

\section{Nuts and bolts}

\subsection{Algebraic objects}

\subsubsection{\tt monomial.py}
This file defines the \texttt{Monomial} class.  Monomial objects
consist of a tuple of pairs (object, exponent) and a coefficient (mod
2). Object can be either a string or integer. If an integer, it is
considered as an indexed variable $\mathrm{Sq}^i$.  Take, for example,
\begin{quote}
\begin{verbatim}
Monomial((("t", 2), (3, 1), ("p", 1)), 1)
\end{verbatim}
\end{quote}
which, if interpreted in the mod 2 Steenrod algebra over $\mathbb{R}$,
is $\tau^2 (\mathrm{Sq}^3)^1 \rho^1$.

The most important methods of the class are {\tt collect\_terms,
  expand\_terms, left\_multiply, right\_multiply}. See the docstrings
for more on the use of these methods.

\subsubsection{\tt polynomial.py}
This file defines the \texttt{Polynomial} class. Polynomial objects
consist of a tuple of monomial objects. The object represents the sum
of the monomial objects in the tuple.

Methods are defined so that polynomial objects can be added with the
``+'' binary operation, e.g., if $A$ and $B$ monomial objects, then
{\tt Polynomial((A)) + Polynomial((B))} returns the object {\tt
  Polynomial((A, B))}. Similarly, multiplication is defined, and the
binary operator ``*'' can be used to multiply polynomials.

\subsubsection{ {\tt r\_monomial.py}}
The {\tt RMonomial} class defines specific Monomial objects which are
custom tailored to represent monomials in the motivic Steenrod
algebra. 

Of great importance is the method {\tt case\_modification} which will
update a monomial to reflect the structure of motivic cohomology of
the given field. (If you want to add your own field, you need to add
the correct procedure for your field.)

{\tt RMonomial} objects are used to represent both elements in the
motivic Steenrod algebra and the dual motivic Steenrod
algebra. Several methods are given for changing notation
appropriately, e.g., {\tt dualize, make\_dual, make\_standard}.

When this class is used in the motivic Steenrod algebra, the method
{\tt is\_admissible} is used in getting normal forms for monomials.

\subsubsection{ {\tt r\_polynomial.py}}
The {\tt RPolynomial} class is a specific instance of the Polynomial
object which is custom tailored for use to represent general members
of the motivic Steenrod algebra. The defining tuple of monomial
objects should consist of {\tt RMonomial} objects.

This class implements methods to use the Adem relations and reach a
canonical form for any element of the motivic Steenrod algebra. As
this is a rather computationally intensive process, as the program
discovers relations in the motivic Steenrod algebra it stores them in
a database for future use. There are three database object types which
are supported: pickled databases, shelf storage, and memory
storage. The pickled storage has the advantage that the relations are
saved for future sessions, which speeds up the user interface. 

Given an RPolynomial object {\tt X}, simply run {\tt X.simplify()} to
mutate {\tt X} into canonical form.

The Adem relations and commutation relations involving the motivic
cohomology of the field are also coded up in this file. These will
need to be modified for fields with mod 2 motivic cohomology different
from $\mathbb{R}$, $\mathbb{C}$, $\mathbb{F}_q$.

\subsubsection{ {\tt r\_steenrod\_algebra.py}}

This file defines helper functions to produce RPolynomial and
RMonomial objects.{\tt RSq} takes in an arbitrary number of arguments
which are either natural numbers or elements of the motivic cohomology
of the base field, e.g., {\tt "t", "p", "u"}. The command {\tt
  RSq("p",2,3,"t")} defines the RPolynomial object corresponding to
$\rho \Sq^2\Sq^3\tau$ in the motivic Steenrod algebra over a field.
The command {\tt RSq()} returns the unit in the motivic Steenrod
algebra.

\subsubsection{ {\tt stack\_obj.py}}

This file defines a class for a stack object used in the methods to
simplify an element of the motivic Steenrod algebra, i.e., an
RPolynomial. 

\subsection{Storage objects}

\subsubsection{ {\tt pickle\_storage.py}}

The program is set up to expect storage objects, as declared in the
{\tt Options} class for a given session. Storage objects need to have
the following methods: {\verb|__iter__,|} {\tt next, copy, read,
  contains, write}.

{\tt PickleStorage} is one instance of a storage object, and the
default choice used by the program. {\tt PickleStorage} uses a python
manager to control access to a dictionary. This allows
multiprocessing, which speeds up the procedures for simplifying
RPolynomial objects.

\subsubsection{ {\tt shelf\_storage.py}} 

{\tt ShelfStorage} is another storage object which utilizes shelves as
the method of storage. Shelf storage is a bit slow compared to pickle
storage or memory storage.

\subsubsection{ {\tt mem\_storage.py}}

{\tt MemStorage} is another storage object which just keeps everything
in memory. It is faster than {\tt PickleStorage} or {\tt
  ShelfStorage}, but it does not save the information generated for
future sessions.

\subsection{Linear algebra}

\subsubsection{{\tt vector\_space.py}}
The {\tt ModTwoVectorSpace} object consists of a list of basis
elements.

\subsubsection{{\tt bit\_vector.py}}
The class {\tt BitVector} enables efficient mod 2 linear algebra to
take place. The static method {\tt random\_vector} is useful when
testing methods. A random vector of lenght {\tt length} is obtained
by{\tt x = BitVector.random\_vector(length)}. Methods for addition, dot
products, printing, and pickling are included. 

A {\tt BitVector} object uses arrays to store a sequence of integers
whose description in base 2 corresponds to the desired vector. What follows is some sample code of how to use bit vectors. 

\begin{quote}
\begin{verbatim}
from bit_vector import *
x = BitVector(5); y = BitVector(5)
z = x + y
print x, y, z
x[2] =1, y[4]=1
z = x+y
print x, y, z
print x*y, x*z
t= BitVector.random_vector(5)
print t*x, t*y, t*z
\end{verbatim}
\end{quote}

\subsubsection{{\tt bit\_matrix.py}} 

{\tt BitMatrix} objects are lists of {\tt BitVectors} of the same
length. The static methods {\tt identity\_matrix(n),
  get\_blank\_matrix(m,n), get\_random\_matrix(m,n)} return the
$n\times n$ identity matrix, the $m\times n$ $0$ matrix, and a random
matrix of size $m \times n$ respectively.

The {\tt BitMatrix} class as methods to calculate the reduced row
echelon form of a matrix. The reduced row echelon form and the change
of basis matrix are saved by the objcet to reduce computing
time. There are several methods available for finding the inverse of a
matrix, performing transpose, addition, multiplication, and solving
linear equations.

\subsubsection{{\tt cohomology.py}}

A {\tt Cohomology} object consists of two bit matrices $B$ and
$A$. The assumption is that the matrices fit into the cochain complex
\begin{equation*}
  C^{-1} \xrightarrow{A} C^0 \xrightarrow{B} C^{1},
\end{equation*}
that is, the product $B*A$ is defined and is zero. 

The primary method for this class is {\tt get\_cohomology()}. This
command will return a {\tt ModTwoVectorSpace} object with basis
consisting of {\tt BitVector} objects. This is just a particular
choice of representatives for a particular basis of $ker(B)/im(A)$.

\subsection{Homological algebra}

\subsubsection{{\tt generator.py}}

A {\tt Generator} object consists of a name (which is a string), and a
degree. Generator objects are used to define free modules over the
motivic Steenrod algebra. 

\subsubsection{{\tt free\_A\_module.py}}

The file {\verb|free_A_module.py|} contains several class definitions
to make it possible to work wtith free modules over the motivic
Steenrod algebra. A {\tt ModuleMonomial} consists of an {\tt
  RPolynomial} object and a {\tt Generator} object. So, for example,
\begin{verbatim}
h = Generator("h1", (1,1))
x = ModuleMonomial(RSq(1), h)
\end{verbatim}
defines $x = \Sq^1 h_1$ which has degree 2 and weight 1.

The class {\tt ModuleElement} is used to represent a general element
in a free module over the motivic Steenrod algebra. The most important
method is {\tt canonical\_form(options)} which gives the canonical form
of a module element with respect to a fixed basis of the free module. 

Wrapper classes are {\tt ModElt} and {\tt ModEltList}. The first takes
as arguments any number of {\tt ModuleMonomial} objects, and the
latter accepts a list of {\tt ModuleMonomial} objects.

A free module over the motivic Steenrod algebra is defined using the
class {\tt FreeAModule}. A {\tt FreeAModule} object is specified by a
list of generator objects and degree bounds. For degree bounds
$(x,y)$, the free module $M$ is determined for all bi-degrees
$M^{a,b}$ with $-x \leq a \leq x$ and $-y \leq b \leq y$. 

The information about the module is stored in {\tt self.array}. This
is a dictionary whose keys are tuples $(x,y)$ corresponding to the
bi-graded piece, with value a {\tt ModTwoVectorSpace}. There are
methods to generate the array, i.e., determine the mod 2 module
structure of the free module in the given bounds. 

In each graded piece, one needs to switch back and forth between
vector notation and the names in terms of the motivic Steenrod algebra
and the generators. The methods {\tt element\_from\_vector}, and {\tt
  vector\_from\_element} accomplish this.

At the end of the file are functions which will calculate the motivic
Steenrod algebra $\calA^{a,b}$ for bidegrees $a\leq x$ and $b\leq y$
for the given degree bounds $(x,y)$. This is very time consuming, so
it is saved for future use. There is a lot of room for improvement
here!

\subsection{{\tt module\_map.py}}

\subsubsection{{\tt h\_star.py}}
\subsubsection{{\tt h\_dual\_module.py}}
\subsubsection{{\tt E2\_page.py}}
\subsubsection{{\tt plot\_page.py}}
\subsubsection{{\tt ./db/etwo.py}}
\subsubsection{{\tt wrapper.py}}




\end{document}